\section{Information page}
\subsection*{Students}
\begin{tabular}{ll}
   & Arponen Jani \\
   & Ogenda Dancun \\
   & Ruley Brian \\
   & Varis Leo \\
\end{tabular}

\subsection*{Official Instructor}
\begin{tabular}{ll}
   & Sarolahti Pasi \\
\end{tabular}


% \subsection*{Approval}
% The Instructor has accepted the final version of this document
% Date: xx.2.2020

\subsection*{Changelog}
\begin{table}[!h]
\small{
\begin{tabular}{l|l|l|l}
\textbf{Version} & \textbf{Date} & \textbf{Author} & \textbf{Description} \\
\hline
0.1 & 2020-08-26 & All & Template \\
0.9 & 2020-08-27 & Jani & Majority written. TODO: Dancun info on GUI.\\
0.9.1 & 2020-08-27 & Jani & Requirement B5 was implemented, changed to Yes \\
1.0 & 2020-08-27 & Jani and Dancun & GUI info \\
1.1 & 2020-08-27 & Jani & Added external library info \\
1.2 & 2020-08-28 & Jani & Added notes about cmake in build and Valgrind errors in tests.\\
1.3 & 2020-08-28 & Jani & Added notes about build warnings. \\
\end{tabular}
}
\caption{Document changelog.}
\label{table:changelog}
\end{table}


\newpage 
\tableofcontents

\newpage
\section{Introduction} 

The purpose of this document is to be the final documentation of the project in the \textit{ELEC-A7151 - Object oriented programming with C++} course. The output of the project was a network simulator  where it is possible to build simple networks and simulate traffic between nodes. The project plan outlined the features that we aimed to implement:

\begin{table}[!htbp]
\footnotesize{
\begin{tabular}{p{0.11\textwidth}|p{0.06\textwidth}|p{0.6\textwidth}|p{0.11\textwidth}}
\textbf{Module} & \textbf{Req\#} & \textbf{Requirement} & \textbf{Implemented} \\
\hline Compatibility
& S1        & It shall be possible to compile and run the program on Ubuntu 18.04. & Yes\\
& S2        & A CMake file shall be provided together with the source code upon completion. & Yes\\
\hline Network model   
& B1        & The network shall be modeled by nodes and links between nodes. Communication between nodes shall be done by (data) packets over the links.  & Yes \\
& B1.1      & Links shall be defined by a transmission speed and a propagation delay, which shall govern: 1. how fast new packets can be sent; and 2. how fast they propagate over the link. There shall be a way to queue packets at the node before the link. & Yes \\
& B1.2      & Nodes shall be defined by an address and are of a type: router or end-host. & Yes\\
& B1.2.1    & Routers shall be able to route packets between other nodes. & Yes\\
& B1.2.2    & End-hosts shall be able to run applications that can send and/or receive packets to/from other end-hosts for a specified length of time. & Yes\footnote{}\\
& B2        & The model code shall be written in such a way that it is easy to extend with e.g. new kinds of links or applications. & Yes\\ 
\hline Program
& B3        & Running simulations shall be "easily configurable" for different network scenarios, through e.g. configuration files. & Yes\\
& B4        & It shall be possible to collect statistics on the simulated network, e.g. packet to destination times, link utilization, queue lengths, etc. & No\\
& B5        & From the applications user interface, it shall be possible to follow the progress of simulation, including statistics and states for links, queues and packets. & Yes \\
\hline GUI
& A1        & There shall be a graphical user interface (GUI) for the program to interact with all other functionalities. & No\\
& A1.1      & B5 shall be expanded to an animation on the GUI. & No\\
\hline Expanded functionality
& A2        & B1.1 shall be expanded to create different queue behaviours, including limited queues and as a result, dropped packets. & No\\
& A3        & B1.1 and B1.2 shall be expanded to include mobile hosts, i.e. wireless links. & No\\
& A3.1      & Communication parameters of wireless links shall be defined by signal strength. & No\\
& A3.2      & Mobile hosts shall be able to move around in a 2D map with obstacles that reduce the signal strength of the wireless link. & No\\
\hline
\end{tabular}
}
\caption{The programs functional requirements and implementation status. (1: Only one application was implemented.) }
\label{table:requirements}
\end{table}

\section{Software structure}
The architecture is very straight forward and shown in figure \ref{img:classdiagram}. Note that the diagram doesn't show absolutely everything, e.g. some getters and setters are excluded together with unimportant data types and some helper methods. An explanation of each classes role in the architecture follows below. 
\\\\
The root class is \texttt{Simulatable}, which handles the simulations timesteps through \texttt{evt\_times[]} and \texttt{AdvanceTime()}. Each simulatable point in the inherited classes has an entry in \texttt{evt\_times[]}, which basically determines how many timesteps to go until it can be executed. A value of -1 indicates nothing to simulate and once every \texttt{evt\_times[]} entry of every \texttt{Simulatable} is equal to -1, the simulations is ran.
\\\\
The \texttt{NetworkInterface} class handles the IP address and every \texttt{Node} holds one. Initially the plan was for a node to be able to hold multiple network interfaces and a full LAN / WAN implementation for routing packets.
\\\\
The \texttt{Packet} class is an arbitrarily simplified TCP packet, holding the important (to our simulation) headers, such as source and target addresses and a time-to-live value, which gets reduced by 1 on each routing event.
\\\\
The \texttt{Node} class inherits from \texttt{Simulatable} and has multiple functions. For instance, it holds a reference to all neighboring nodes and which links are used to connect to those in the \texttt{connected[]} variable. Packets are handled in separate \texttt{receive[]} and \texttt{transmit[]} queues, the latter of which also holds a reference to which currently connected node the packet is to be transferred to. Multiple helper methods exist, e.g. for connecting or disconnecting from neighboring nodes and moving packets. The \texttt{RunApplication()} method is overloaded by inherited classes to implement application specific functionality.
\\\\
The \texttt{Link} class inherits from \texttt{Simulatable} and is used to link nodes together. The main parameters are \texttt{transmissionspeed} and \texttt{propagationdelay}, which ultimately result in the timestamp written back to \texttt{Simulatable.evt\_times[]}. The link also holds two \texttt{transmissionqueue} variables, which are the currently in transfer packets.
\\\\
The \texttt{EndHost} class inherits from \texttt{Node} and is the node that ultimately generates packets that will be sent to another endhost. The implemented functionality simply generates random sized packets (amount equal to \texttt{packetcount}) with the specified \texttt{targetadr}. If the target address is set to self (as it is by default) then no packets are generated. 
\\\\
The \texttt{Router} class inherits from \texttt{Node} and is the node responsible for routing packets between other nodes. The way this is done, is by going through the \texttt{Node.receive} queue and looking up the self, target address in a routing table generated by the \texttt{Network}. If no route exists, the packet is dropped.
\\\\
The \texttt{Network} class ties everything else together. It is the interface to creating and removing nodes, linking and unlinking them and other graph behaviour. The \texttt{routingtable} map is generated by running Dijkstra's shortest path algorithm from every source node to every target node. The \texttt{Network} also holds the functionality to \texttt{Save} and \texttt{Load} JSON files which contain the network configuration.

\begin{figure}[!htbp]
\begin{center}
	\includegraphics[width=13cm]{classdiagrams.png}
    \caption{Network simulator class diagrams.}
	\label{img:classdiagram}			% use \ref{img:example} to refer to this figure
\end{center}
\end{figure}

\subsection{External libraries}
The GUI portion and the JSON parser are both external libraries provided by Qt. As the GUI was ultimately dropped, only the file saving and loading requires external libraries. More information in section \ref{sec:dependencies}.

\section{Build instructions}

\begin{enumerate}
    \item Clone the project repository to your machine, or download and unzip.
    \item Open terminal and navigate to the project root, run \texttt{cd src/nwsim}
    \item Build the software by first generating the Makefile with \texttt{cmake CMakeLists.txt} and then running \texttt{make}
    \begin{enumerate}
        \item If you get Qt related errors running \texttt{cmake}, see \ref{sec:dependencies}.
    \end{enumerate}
    \item The software is built to the projects root/bin and the executables name is \texttt{nwsim-cli}
\end{enumerate}

\subsection{Dependencies} \label{sec:dependencies}
Building the software requires the \texttt{Qt5::Core} library from Qt. The location of this library depends on your setup, but we have included the following possible paths (below). If running \texttt{cmake} produces library related errors for you, please include the correct path\footnote{Apparently on Aalto Linux machines the Qt libraries are found under \texttt{~/usr/lib} (?)} in the \texttt{src/nwsim/CMakeLists.txt} file. 
\lstset{basicstyle=\scriptsize,numbers=left}
\begin{lstlisting}[firstnumber=3]
# Appends the location of qt5 to CMAKE_PREFIX_PATH **UNIQUE TO HOST**
list(APPEND CMAKE_PREFIX_PATH "~/usr/lib/")
list(APPEND CMAKE_PREFIX_PATH "~/Qt/5.15.0/gcc_64/lib/cmake")
\end{lstlisting}

\subsection{Build warnings}
The networks save to / load from file functionality is developed with Qt's libraries. The original development was done with version 5.12.8, and building with these produces no warnings nor errors. However, building with the latest version (5.15.0) the \texttt{QJsonDocument::FromBinaryData(...)} method is deprecated: \begin{footnotesize}{\url{https://doc.qt.io/qt-5/qjsondocument-obsolete.html#fromBinaryData}}\end{footnotesize}. The reason is simply obsoletion, and not functional. As such we felt that this warning was not necessary to fix.

\section{Running the software}
The software is a CLI implementation and has rather intuitive usage. For usage, please run the \texttt{help} command, or see the compiled version below in section \ref{sec:help}.

\subsection{Example on running the simulator}
\begin{enumerate}
    \item Open a terminal and navigate to the project root. Run the simulator with \texttt{./bin/nwsim-cli}
    \item At any point, type \texttt{help} for usage.
    \item Create an endhost node with \texttt{add e 1.1.1.1}, you will enter edit mode for this node:
    \begin{enumerate}
        \item Run \texttt{list} to see all changeable parameters.
        \item \texttt{exit} or \texttt{quit} or \texttt{q} to drop out of editmode.
    \end{enumerate}
    \item Create a router node with \texttt{add r 2.2.2.2}, drop out of editmode.
    \item Create another endhost node with \texttt{add e 3.3.3.3}, in editmode configure:
    \begin{enumerate}
        \item \texttt{set target 1.1.1.1} to set target to the first endhost.
        \item \texttt{set count 100} to send 100 packets from this endhost.
    \end{enumerate}
    \item Edit the first endhost with \texttt{edit 1.1.1.1} and configure:
    \begin{enumerate}
        \item \texttt{set target 3.3.3.3}
        \item \texttt{set count 50}
    \end{enumerate}
    \item Link the first endhost to the router with \texttt{link 1.1.1.1 2.2.2.2}, and enter links editmode:
    \begin{enumerate}
        \item Run \texttt{list} to see all changeable parameters.
        \item \texttt{set ts 100} to set transmission speed to 100 timeunits.
        \item \texttt{set pd 2} to set propagation delay to 2 timeunits / byte.
        \item \texttt{exit} or \texttt{quit} or \texttt{q} to drop out of editmode.
    \end{enumerate}
    \item To finish the network, link the remaining endhost and the router \texttt{link 2.2.2.2 3.3.3.3} and configure:
    \begin{enumerate}
        \item \texttt{set ts 50} 
    \end{enumerate}
    \item If you created unnecessary nodes, they can be removed with \texttt{rem <adr>}
    \item If you linked the wrong nodes, they can be disconnected with \texttt{unlink <adr> <adr>}
    \item If you wish to edit the links parameters, run \texttt{edit <adr> <adr>}
    \item Run \texttt{list} to see the current network configuration.
    \item Enter the simulation mode with \texttt{sim}
    \begin{enumerate}
        \item Run \texttt{list} to see all endhosts that will be sending packets during the simulation.
        \item Run \texttt{routes} to see the routing table and check that your network isn't missing a crucial link.
        \item Start the simulation with the \texttt{run} command. You will go back to the simulation mode once the simulation has finished.
        \item To exit simulation mode, use \texttt{exit} or \texttt{quit} or \texttt{q}
    \end{enumerate}
    \item To save the current network configuration, run \texttt{save <filename>}
    \item To load a new network configuration from a file, run \texttt{load <filename>}, NOTE: this will overwrite the current configuration.
    \item To exit the program, use \texttt{exit}
\end{enumerate}

\subsection{Help documentation} \label{sec:help}
\lstset{basicstyle=\scriptsize,numbers=no}
\begin{lstlisting}
LEGEND:
<mode>      - edit, sim
<filename>  - JSON file to save/load network config
<adr>       - IP address in octet format, e.g. 192.168.0.1
<int>       - integer, note that parameters are capped between valid values.
NWSIM HELP:
help                - Print this manual.
       <mode>       - Prints mode specific usage.
save   <filename>   - Save current network configuration to given file. Possible file
                      formats: .json .dat
load   <filename>   - Discard current network configuration and load from specified file.
                      Possible file formats: .json .dat
exit                - Exit current mode or if at root, exits the program.
list                - Lists all current nodes and what other nodes they are linked to.
add    e|r <adr>    - Adds an [e]ndhost or [r]outer with given address.
                      NOTE: Address must be IP format and unique in network.
rem    <adr>        - Removes an endhost or router that matches the given address, and
                      severs affected links. If node doesn't exist, nothing happens.
link   <adr> <adr>  - Links given nodes, if they exist.
unlink <adr> <adr>  - Unlinks given nodes, if they are currently linked.
edit   <adr>        - Enter node (endhost, router) edit mode.
edit   <adr> <adr>  - Enter link edit mode.
sim                 - Enter simulation mode.
tests               - Prints result of all tests and exit program.
EDIT MODE HELP:
help                - Print this help for edit mode.
exit                - Exit edit mode.
list                - List all changeable parameters.
NODE SPECIFIC EDIT MODE HELP:
set   address <adr> - Changes this nodes to use the given address.
                      NOTE: Address must be IP format and unique in network.
ENDHOST SPECIFIC EDIT MODDE HELP:
set target <adr>    - Requires endhost source and endhost target. Sets target address.
                      NOTE: Address must be IP format and exist in current network.
                            If set to self, no packets sent in simulation.
set count <int>     - Requires endhost. Sets amount of packets sent to target.
LINK SPECIFIC EDIT MODE HELP:
set ts <int>        - Sets links transmission speed to given value (timeunit). Value 
                      determines the interval at which new packets can be transmitted
                      to the link.
set pd <int>        - Sets links propagation delay to given value (timeunit / byte).
                      Value determines the time it takes for a packet to travel across
                      the link. time = propagation_delay * packet_size
SIM MODE HELP:
help                - Print this help for sim mode.
list                - Lists all endhosts that are configured to send packets.
routes              - Prints current network routing table.
run                 - Starts simulation.
\end{lstlisting}

\section{Tests}
The software was tested while it was being written and refactored, by writing unit tests and updating as necessary when the underlying logic changed. The \texttt{src/nwsim/tests/testroutines.hpp} file holds all testcases and they can be executed in software by running the \texttt{tests} command. The program output was then simply searched for the flag \texttt{false} for any tests that didn't pass, e.g:
\begin{enumerate}
    \item \texttt{./bin/nwsim-cli | tee testout.txt}
    \item Run the \texttt{tests} command for test output and \texttt{exit}
    \item Search for tests that didnt pass with \texttt{grep false testout.txt}
\end{enumerate}

\subsection{Test output}
All tests that pass hold value \texttt{true}. If it was not possible to write simple and quick enough tests, e.g. for the routing table or actual simulation output, then a manual look is needed. In the simulation test output not all packets have moved to their corresponding nodes. This is because the test is only run for \texttt{n} timesteps, and the packet counts are too high to be transferred during this period. 

\subsection{Valgrind}
Running the network simulator with valgrind shows, that there is some memory leak still present. This was introduced in merging the feature of saving / loading the network to / from a file. After spending (too) many hours hunting this issue, we have decided to leave it as is.
\lstset{basicstyle=\scriptsize,breaklines=true}
\lstinputlisting{noleak-commit-1e911efa9b.txt}
\lstinputlisting{latest.txt}



\section{Work log}
The amount of hours for each week for every group member is described here.

\begin{table}[!htbp]
\footnotesize{
\begin{tabular}{c|p{0.15\textwidth}|c|p{0.55\textwidth}}
\textbf{Week} & \textbf{Dates} & \textbf{Hours} & \textbf{What was done?} \\
\hline 1 & July 6 - July 12     & 2 & Project plan \\
\hline 2 & July 13 - July 19    & 3 & Weekly meeting, node, link and network design. \\
\hline 3 & July 20 - July 26    & 11 & Weekly meeting, studying CMake and Qt Creator. initial work on IP address handling helpers, NetworkInterface, Packet, Node, Link and Application classes. Test cases for NetworkInterface and Packet.  \\
\hline 4 & July 27 - Aug 2      & 6 & Weekly meeting. Continued work on Node and Link classes. Tests for Nodes, Links and initial packet transfer. Node position variables + tests for Dancun. \\
\hline 5 & Aug 3 - Aug 9        & 2 & Weekly meeting. Mid-term meeting with Pasi. \\
\hline 6 & Aug 10 - Aug 16      & 0 & Nothing \\
\hline 7 & Aug 17 - Aug 23      & 25 & Weekly meeting. Major refactor and cleanup of Node and Link classes, dropped Application as a separate class. Network class to hold Nodes and Links, routing, EndHost, Router, Packet generation. Testcases for Node, Link, Network. Moved testcases to separate file and introduced assert checks to grep for fails.  \\
\hline 8 & Aug 24 - Aug 30      & 40 & Weekly meeting. Simulatable nodes and links, packet transfer between them, fix routing, testcases. Wrote CLI as GUI was abandoned. Final documentation. Reviewing file save/load implementation from Brian. Demo with Pasi and group 1. Hunting for memory leak in file save/load. Integrating to CLI. \\
\hline\hline\textbf{Total}&All  & 99 & - \\
\end{tabular}
}
\caption{Jani's work log. Hours are rough estimates from git commit history, stopped tracking on week 7 to keep up.}
\label{table:worklog-jani}
\end{table}

%TODO: Dancun
\begin{table}[!htbp]
\footnotesize{
\begin{tabular}{c|p{0.15\textwidth}|c|p{0.55\textwidth}}
\textbf{Week} & \textbf{Dates} & \textbf{Hours} & \textbf{What was done?} \\
\hline 1 & July 6 - July 12     & 4 & Studied about Qt and how to use it in making GUIs in C++ \\
\hline 2 & July 13 - July 19    & 2 & Tried making nodes which would represent the routers plus the necessary buttons needed to add nodes \\
\hline 3 & July 20 - July 26    & 4 & Researched on a ways in which to connect the nodes to each other so that traffic could be implemented in the network between the nodes \\
\hline 4 & July 27 - Aug 2      & 2 & Changed the GUI as the original setup I had was not able to send any communication over the kink between the two nodes \\
\hline 5 & Aug 3 - Aug 9        & 1 & Worked on improving the GUI by adding the needed buttons and other features needed \\
\hline 6 & Aug 10 - Aug 16      & 2 & Came with nodes that can display on the screen and can be connected to each other \\
\hline 7 & Aug 17 - Aug 23      & 2 & Tried adding the source code but did not find the expected results on how to tie the GUI to the code \\
\hline 8 & Aug 24 - Aug 30      & 1 & GUI works but still problems trying to get the code and GUI to work \\
\hline\hline\textbf{Total}&All  & 18 & - \\
\end{tabular}
}
\caption{Dancun's work log. }
\label{table:worklog-dancun}
\end{table}


\begin{table}[!htbp]
\footnotesize{
\begin{tabular}{c|p{0.15\textwidth}|c|p{0.55\textwidth}}
\textbf{Week} & \textbf{Dates} & \textbf{Hours} & \textbf{What was done?} \\
\hline 1 & July 6 - July 12     & 4 & UML for Project Plan, studied network basics \\
\hline 2 & July 13 - July 19    & 1 & Weekly meeting. \\
\hline 3 & July 20 - July 26    & 5 & Weekly meeting. Studied the use of CMake and linking libraries \\
\hline 4 & July 27 - Aug 2      & 3 & Weekly meeting. Decided on the use of Qt's own JSON libraries. \\
\hline 5 & Aug 3 - Aug 9        & 4 & Weekly meeting. Figured out how to use QJson-classes, some preliminary implementations for reading/writing to JSON.\\
\hline 6 & Aug 10 - Aug 16      & 0 & Nothing \\
\hline 7 & Aug 17 - Aug 23      & 2 & Weekly Meeting. \\
\hline 8 & Aug 24 - Aug 30      & 5 & Implemented save/load functionality for user-configurable attributes of the network in JSON file format. \\
\hline\hline\textbf{Total}&All  & 24 & - \\
\end{tabular}
}
\caption{Brian's work log. }
\label{table:worklog-brian}
\end{table}

\begin{table}[!htbp]
\footnotesize{
\begin{tabular}{c|p{0.15\textwidth}|c|p{0.55\textwidth}}
\textbf{Week} & \textbf{Dates} & \textbf{Hours} & \textbf{What was done?} \\
\hline 1 & July 6 - July 12     & 3 & Project plan, studied network protocols \\
\hline 2 & July 13 - July 19    & 2 & Attended weekly meeting, planned parts of the implementation \\
\hline 3 & July 20 - July 26    & 2 & Attended weekly meeting, wrote initial version of event queue \\
\hline 4 & July 27 - Aug 2      & 2 & Attended weekly meeting, rewrote event queue \\
\hline 5 & Aug 3 - Aug 9        & 1 & Attended weekly meeting \\
\hline 6 & Aug 10 - Aug 16      & 1 & Attended weekly meeting \\
\hline 7 & Aug 17 - Aug 23      & 3 & Attended weekly meeting, created routing table for the network \\
\hline 8 & Aug 24 - Aug 30      & 0 & nothing \\
\hline\hline\textbf{Total}&All  & 14 & - \\
\end{tabular}
}
\caption{Leo's work log. }
\label{table:worklog-leo}
\end{table}
\hfill
\section{Post mortem}
It is quite obvious from the output of the project that we did not meet the goal that we set for ourselves and what was expected from the project description. There are many reasons for why this was the case, but the core issues were:
\begin{enumerate}
    \item Work was probably divided wrong from the start, e.g. Jani had work until the last 2 weeks of the project, but everyone else was waiting for the results of his code. Similarly, Leo had free time at the start, but less towards the end. See figure \ref{img:schedule} for the original schedule from the project plan.
    \item Design of the software was only done superficially and at a too high level. The approach was to design further as we progressed, but this turned out to be a bad decision. Ad-hoc approach to design rarely works, and here it didn't either.
    \item Related to the design, but the architecture was started from the bottom up, which introduced some unnecessary issues, which could have been averted if the network was immediately designed as a graph and the necessary interfaces implemented early, such that other group members wouldn't have to worry about ever-changing code.
    \item The GUI portion was started early and progressed fine, but it was never properly integrated with the core functionality. This is why, in the end, a CLI approach was taken as there was not enough time left to write a shorter implementation and integration. 
\end{enumerate}

\begin{figure}[!htbp]
\begin{center}
	\includegraphics[width=13cm]{originalschedule.png}
    \caption{Original schedule from the project plan.}
	\label{img:schedule}			% use \ref{img:example} to refer to this figure
\end{center}
\end{figure}

\subsection{GUI}
The GUI was being worked on as mentioned, and some functionality was achieved:
% TODO Dancun
\begin{enumerate}
    \item The GUI we have so far has the ability to add nodes on the screen. The nodes can be connected to each other and a given node can be deleted. The implementation is simple and does not include 2D or 3D
    \item For the time being the functionality that exist, is one to add nodes and connect them. There is not any additional functionalities that have been implemented as of now
\end{enumerate}

\begin{figure}[!htbp]
\begin{center}
	\includegraphics[width=8cm]{Screenshot 2020-08-27 at 14.03.23.png}
    \caption{Screenshot 1 from the GUI.}
	\label{img:gui-1}			% use \ref{img:example} to refer to this figure
\end{center}
\end{figure}

\begin{figure}[!htbp]
\begin{center}
	\includegraphics[width=8cm]{Screenshot 2020-08-27 at 14.03.42.png}
    \caption{Screenshot 2 from the GUI.}
	\label{img:gui-2}			% use \ref{img:example} to refer to this figure
\end{center}
\end{figure}

